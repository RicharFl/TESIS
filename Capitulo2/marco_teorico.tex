
%%%%%%%%%%%%%%%%%%%%%%%%%%%%%%%%%%%%%%%%%%%%%%%%%%%%%%%%%%%%%%%%%%%%%%%%%
%           Capítulo 2: MARCO TEÓRICO - REVISIÓN DE LITERATURA
%%%%%%%%%%%%%%%%%%%%%%%%%%%%%%%%%%%%%%%%%%%%%%%%%%%%%%%%%%%%%%%%%%%%%%%%%

\chapter{Estado del arte}
En este apartado, se presentan las tecnologías y trabajos que guardan relación al sistema que se plantea desarrollar, ya sea por tema o el uso de tecnologías seleccionadas e implementadas en cada uno de ellos.

\section{Software de Mesas de Servicios en el mercado}

\subsection{BMC Helix ITSM}

BMC Helix ITSM es una solución potente y centrada en las personas que aprovecha las tecnologías emergentes, tales como la IA y el aprendizaje automático. Cuando se cambia de Remedy en las instalaciones a BMC Helix ITSM [2] , obtiene lo siguiente:

\begin{itemize}
	 
	\item Gestión predictiva de servicios mediante la clasificación, la asignación y el enrutamiento automáticos de las incidencias
    \item Competencias incorporadas de varias nubes para intermediar incidencias, cambios y versiones a través de los proveedores de nube
	\item Integración con las principales herramientas ágiles de DevOps como Jira
	\item Análisis de correo electrónico cognitivo y acciones automatizadas en nombre del usuario
	\item Eficiencias operacionales y de despliegue mediante el uso de contenedores
   \item Alineación con ITIL V4 
	
\end{itemize}

\subsection{Remedy Service Desk de BMC}

Remedy Service Desk es una aplicación de software para implementar una mesa de ayuda a nivel enterprise compatible con los procesos de ITIL que incluye manejo de indicentes, manejo de problemas, niveles de servicio y muchas facilidades más ya que forman parte de la suite BMC Remedy ITSM. Remedy Service Desk es la aplicación líder de la industria y ahora con la versión de Remedy 9, con una gran versatilidad con dispositivos móviles, apps e interfaces sociales. [3]

La aplicación implementa las funciones de mesa de ayuda permitiendo operar un único punto de contacto entre los usuarios y el área de TI. Remedy Service Desk cumple con los procesos de ITIL y ofrece los siguientes beneficios:

\begin{itemize}	
\item	La única mesa de servicio con visibilidad directa a los problemas del negocio.
\item	Flexible, escalable y modular con un poderoso motor de flujos de trabajo.
\item	Automatiza los procesos de la mesa de servicios.
\item	Interfaz estándar con toda la suite BMC Remedy ITSM
\item	Incorporación de procesos y reglas de negocio
\item	Soporta múltiples empresas y múltiples idiomas.
	Basada en roles
\item	Multicapas y multiplataforma
\item	Construida con base a las Mejores Prácticas de ITIL
\item	Accesos vía web, correo, PDAs, etc.
\item	Explotar la información en forma fácil y en tiempo real
\item	Manejar múltiples mesas de ayuda (multi tenancy)
\item	El mejor ROI de la industria
\item	Cuenta con su propia herramienta de desarrollo (ARSystem)
\item	Completamente compatible con ITIL
	
\end{itemize}

 \subsection{Aranda ASDK}
 
Complementa las funcionalidades de ASDK, adquiriendo diferentes soluciones de Aranda que facilitan la gestión de sus recursos: Integración con Aranda CMDB (Configuration Management Database), logrando una solución que de acuerdo con las mejores prácticas de ITIL se denomina Configuration Management. Esta fusión permite asociar los elementos de configuración (CI’s) relacionados con los procesos de soporte, a los incidentes y llamadas de servicio, logrando una gestión completa sobre la infraestructura IT de su organización. Nuestra solución de inventario automatizado de hardware y software Aranda ASSET MANAGEMENT, permite el control y administración remota de sus estaciones de trabajo para agilizar el soporte y reducir considerablemente los tiempos de respuesta al integrarse con ASDK. Integración con Aranda DASHBOARD (ADSB), logrando acceder a una interfaz gráfica para visualizar los indicadores de procesos de soporte como llamadas de servicio, incidentes y problemas, mejorando la gestión de la mesa de servicio. [4]

\section{Tesis y trabajos de investigación relacionados} 

Actualmente existen muchas medianas y pequeñas empresas que operan en el medio, pero carecen de un sistema de información que les de soporte en sus procesos operacionales y muchas veces solicitar un desarrollo a medida es costoso, y en la mayoría de los casos no satisface sus expectativas necesitando un soporte continuo por parte de desarrollador. Se encuentra muchas pymes donde el principal proceso a dar soporte es el de ventas ya que a partir de allí se puede ir creciendo de manera paulatina para ir gestionando luego otros procesos a medida que crezca la organización. 

\begin{itemize}
	\item 1.	En su tesis “implementación de la mesa de servicio aplicando itil v. 3.0 para mejorar la calidad del servicio en la oficina de sistemas de informacion de la universidad privada de la selva peruana, iquitos 2018” [7] , desarrolló la puesta en marcha de una Mesa de Servicio basada en ITIL V 3.0 con el empleo de la aplicación web GLPi (Gestionnaire Libre de Parc Informatique), en la Oficina de Sistemas de Información de la Universidad Privada de la Selva Peruana, gestionando la atención de incidentes para mejorar la calidad del servicio de apoyo técnico del personal encargado. Se logró implementar una mesa de servicio y demostrar la mejora de la calidad del servicio. Se realizaron medidas del nivel de calidad empleando el cuestionario de servicio SERVPERF aplicándolo a los Administrativos (10 personas) y docentes (70) personas. La conclusión del estudio fue demostrar mediante la prueba t de muestras dependientes que la implementación de la mesa de servicio siguiendo ITIL V 3.0 permitió mejoras en la calidad percibida del servicio en los administrativos (17.7%) y en los docentes (15.5%), logrando en ambos casos pasar de un nivel de juicio aceptable a un nivel de juicio bueno.
	
	\item En su tesis “: estrategia, diseño y transición de una mesa de ayuda aplicando itil v3, caso de estudio: tcontrol s.a.” [8] se enfoca el desarrollo de los procesos adecuados para la creación de servicios de la mesa de ayuda, aplicando la metodología ITIL V3 como marco de referencia con la finalidad de brindar calidad en los servicios TI a los usuarios internos. Para medir su nivel actual de madurez del servicio en las fases: estrategia, diseño y transición, se recopiló información mediante diferentes medios, cuyo puntaje fue 1,7 en una escala de 0 a 5, siendo 5 el mayor puntaje, llegando a un nivel de madurez del servicio a 2.8 como resultado final después del desarrollo e implementación de la investigación realizada, siendo este un referente muy importante si en el futuro la empresa desea implementar planes de mejora continua.
	
	\item En su tesis “Creación de una Mesa de Ayuda Basada en ITIL V3 para una Empresa del Sector Minero” [10] consiste en proponer el análisis, diseño e implementación de un sistema de mesa de ayuda basado en ITIL V3, también conocido como Help Desk, para la empresa de servicios generales GEOMAD E.I.R.L., empresa que se encuentra ubicada en el sector minero en la ciudad de Lima, Perú, así mismo la empresa cuenta con una oficina especial en la ciudad de Arequipa y sedes en el sur del país de acuerdo a la demanda de proyectos mineros. El sistema de mesa de ayuda está destinado para atender los incidentes y problemas, los mismos que serán documentados y gestionados en base a las requerimientos que se presenten por parte del personal de la empresa que tendrá como objetivo la satisfacción con los servicios del área de Tecnologías de la Información, asimismo para su correspondiente implementación se realizó por etapas la investigación preliminar, los requerimientos para el sistema, el análisis y diseño del sistema, las pruebas correspondientes, su documentación e implementación para proceder a largo plazo con el mantenimiento del sistema, el cual estará desarrollado en PHP y disponible para cualquier dispositivo tecnológico que tenga conexión a la red de la empresa.
	
	

\end{itemize}


Realizando un análisis con los software  existentes en el mercado se presenta la siguiente Tabla \ref{tab:CMDS-OTHER} la cual refleja la comparativa realizada con entre dichos software y la solución a desarrollar.

  % Table generated by Excel2LaTeX from sheet 'Hoja2'
\begin{table}[htbp]
	\centering
	\caption{Cuadro de comparación software en mercado- solución propuesta}
	\scalebox{0.55}{\begin{tabular}{|c|p{5.355em}|p{5.355em}|p{5.355em}|p{5.355em}|p{5.355em}|}
		\toprule
		\rowcolor[rgb]{ .357,  .608,  .835} \textbf{Identificador} & \multicolumn{1}{c|}{\textbf{PaaS}} & \multicolumn{1}{c|}{\textbf{Servicio Web}} & \multicolumn{1}{c|}{\textbf{Redundancia de base de datos, en una local.}} & \multicolumn{1}{c|}{\textbf{Buenas practicas ITIL }} & \multicolumn{1}{c|}{\textbf{Reporte por correo}} \\
		\rowcolor[rgb]{ .871,  .918,  .965} \textbf{1} & No & No & No & Si & No \\
		\midrule
		\textbf{2} & No & Si & No & Si & No \\
		\midrule
		\rowcolor[rgb]{ .871,  .918,  .965} \textbf{3} & No & No & No & Si & No \\
		\midrule
		\textbf{4} & No & Si & Si & Si & No \\
		\midrule
		\rowcolor[rgb]{ .871,  .918,  .965} \textbf{BMC Helix ITSM} & Si & Si & No & No & Si \\
		\midrule
		\textbf{Remedy Service Desk } & Si & Si & No & Si & SI \\
		\midrule
		\rowcolor[rgb]{ .871,  .918,  .965} \textbf{Aranda SDK} & Si & Si & No & Si & Si \\
		\midrule
		\rowcolor[rgb]{ 1,  .851,  .4} \textbf{Propuesta} & Si & Si & Si & SI & Si \\
		\bottomrule
	\end{tabular}%
	\label{tab:CMDS-OTHER}}%
\end{table}%
