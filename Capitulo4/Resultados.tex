\chapter{Análisis y Diseño}
Para desarrollar un sistema de calidad se necesita requerimientos hechos por el cliente, los cuales están proporcionados, por la PyME solicitante del sistema, la etapa de análisis y diseño estarán desarrolladas por módulos. 

El siguiente análisis estará compuesto por:

\begin{itemize}
	\item Especificación de requerimientos 
	\item 	Modelo de comportamiento
	\begin{itemize}
		\item Modelo de procesos 
		\begin{itemize}
			\item	Diagrama de frujo 
			\item 	Diccionario de datos 
		\end{itemize}
	\item Modelo de datos
	\begin{itemize}
		\item Diagrama de entidad Relación
	\end{itemize}
	
	\end{itemize}
	
		\item Interfaz gráfica de usuario.
\end{itemize}





    \section{Requerimientos  del sistema}
    
   En este sección  contiene los requerimientos que serán necesarios para el correcto funcionamiento del software de mesa de servicio. 
   Los requerimientos solo serán mencionados el secciones posteriores se describirán puntualmente cada uno de ellos.

\subsection{Requerimientos no funcionales }


\begin{itemize}
	\item Portafolio de servicio
	\item Categorización de servicio 
	\item Matriz de urgencia e impacto 
	\item Acuerdos de nivel de servicio (SLA).
	\item Roles y puestos de mesa de servicio.
	\item Seguridad
	\item Disponibilidad
	
\end{itemize}

\subsubsection{Portafolio de servicio}
Actualmente la PyME se encuentra en un planteamiento de apertura de nuevo portafolio de servicios, sin embrago en la actualidad cuenta con las siguientes descripciones de servicios como se puede ver en la tabla \ref{tab:CatSer}.

  % Table generated by Excel2LaTeX from sheet 'Portafolio de servicos '
\begin{table}[H]
	\centering
	\caption{Descripción de portafolio de servicio}
	\scalebox{0.55}{	\begin{tabular}{|p{17.57em}|p{64.715em}|}
		\toprule
		\rowcolor[rgb]{ .122,  .22,  .392} \textcolor[rgb]{ 1,  1,  1}{SERVICIO} & \textcolor[rgb]{ 1,  1,  1}{DESCRIPCIÓN DE SERVICIO} \\
		\midrule
		Administración de servidores & Administración de Centros de Computo e infraestructura tecnológica.\newline{}Instalación, configuración, mantenimiento. correctivo y preventivo.\newline{}Administración de servidores Linux,. Windows Server, Asterisk, directorio activo, correo electrónico.\newline{}Realizamos migraciones de Sistemas. Operativos.\newline{} \\
		\midrule
		Venta y renta de equipo de cómputo. & Venta y renta de equipo de computo laptop, CPU \newline{}Venta y renta de perifericos de equipos de laptop y CPU \newline{}Parner de HP, DELL, LENOVO, APPLE,  \\
		\midrule
		Mesa de servicio  & Al llamar a la mesa de ayuda, el usuario indica su problema y el agente receptará su petición y generará el ticket correspondiente, si se necesita la presencia física del técnico se escalará al siguiente nivel para que el soporte 2 realice la visita técnica reasignando así el ticket de atención. \\
		\bottomrule
	\end{tabular}%
	\label{tab:CatSer}}%
\end{table}%
\subsubsection{Categorización de servicios}
En la siguiente sección se define el  conjunto completo de servicios, los cuales forman parte el portafolio de servicios de la PyME, sin embargo el análisis de la categorización de servicios solo implicara el servicio de “mesa de servicio”. 
La categorización de servicios se realiza en dos segmentos los cuales son:
\begin{itemize}
	\item Incidente
	\item Requerimientos. 
\end{itemize}


En la tabla \ref{tab:INCSOF} podemos observar los incidentes correspondientes a software así como el servicio que proporciona TI a dichos incidentes.




   % Table generated by Excel2LaTeX from sheet 'programas '
 \begin{table}[H]
 	\centering
 	\caption{Incidentes de tipo Software}
 	\scalebox{0.55}{\begin{tabular}{|c|c|l|c|}
 		\toprule
 		\rowcolor[rgb]{ .267,  .447,  .769} \textcolor[rgb]{ 1,  1,  1}{Tipo de servicio} & \textcolor[rgb]{ 1,  1,  1}{Categoria} & \multicolumn{1}{c|}{\textcolor[rgb]{ 1,  1,  1}{Subcategoria}} & \textcolor[rgb]{ 1,  1,  1}{Servicios de IT} \\
 		\midrule
 		\multirow{15}[29]{*}{\begin{turn}{-90}INCIDENTE  \end{turn}} & \multirow{15}[29]{*}{\begin{turn}{-90}PROGRAMAS \end{turn}} & Outlook & \multicolumn{1}{c|}{\multirow{15}[29]{*}{* Instalación\newline{}\newline{}* Activación licencia\newline{}\newline{}* Actualización}} \\
 		\cmidrule{3-3}        &   & Word  &  \\
 		\cmidrule{3-3}        &   & Excel &  \\
 		\cmidrule{3-3}        &   & PowerPoint &  \\
 		\cmidrule{3-3}        &   & Visio &  \\
 		\cmidrule{3-3}        &   & AutoCAD &  \\
 		\cmidrule{3-3}        &   & SAP &  \\
 		\cmidrule{3-3}        &   & Antivirus &  \\
 		\cmidrule{3-3}        &   & SQL server &  \\
 		\cmidrule{3-3}        &   & Acrobat &  \\
 		\cmidrule{3-3}        &   & Java &  \\
 		\cmidrule{3-3}        &   & Visual studio &  \\
 		\cmidrule{3-3}        &   & Office 365 &  \\
 		\cmidrule{3-3}        &   & Google crome  &  \\
 		\cmidrule{3-3}        &   & Internet explore   &  \\
 	\end{tabular}%
 	\label{tab:INCSOF}}%
 \end{table}%

.
En la  tabla \ref{tab:INCHAR}, se describen los incidentes, con una subcategorización de hardware,  donde se describen los servicios proporcionados por el departamento de IT. 

  % Table generated by Excel2LaTeX from sheet 'Harware '
\begin{table}[H]
	\centering
	\caption{Incidentes de tipo hardware}
	\scalebox{0.55}{\begin{tabular}{|c|c|l|c|}
		\toprule
		\rowcolor[rgb]{ .267,  .447,  .769} \textcolor[rgb]{ 1,  1,  1}{Tipo de servicio} & \textcolor[rgb]{ 1,  1,  1}{Categoria} & \multicolumn{1}{c|}{\textcolor[rgb]{ 1,  1,  1}{Subcategoria}} & \textcolor[rgb]{ 1,  1,  1}{Servicios de IT} \\
		\midrule
		\multirow{13}[26]{*}{\begin{turn}{-90}I N C I D E N T E  \end{turn}} & \multirow{13}[26]{*}{\begin{turn}{-90}H A R W A R E \end{turn}} & Laptop / portatil  & \multicolumn{1}{c|}{\multirow{13}[26]{*}{Instalación\newline{}Falla\newline{}Cambio de sitio\newline{}Configuración \newline{}Reposición \newline{}Mantenimiento}} \\
		\cmidrule{3-3}        &   & Pc &  \\
		\cmidrule{3-3}        &   & Monitor &  \\
		\cmidrule{3-3}        &   & Teclado &  \\
		\cmidrule{3-3}        &   & Mouse  &  \\
		\cmidrule{3-3}        &   & Lector de DVD  &  \\
		\cmidrule{3-3}        &   & Impresora &  \\
		\cmidrule{3-3}        &   & Escáner  &  \\
		\cmidrule{3-3}        &   & Disco duro interno /externo   &  \\
		\cmidrule{3-3}        &   & Proyector  &  \\
		\cmidrule{3-3}        &   & Proyector  &  \\
		\cmidrule{3-3}        &   & Pantallas /tv &  \\
		\cmidrule{3-3}        &   & UPS &  \\
		\bottomrule
	\end{tabular}%
	\label{tab:INCHAR}}%
\end{table}%




Dentro del servicio a proporcionar se consideran dos tipos de servicios, soporte técnico y administrativos de control, donde se incluirá dentro de administrativos de control la sub categoría requerimiento la cual se define como todo aquel servicio que no represente una falla.

A continuación en la siguiente tabla \ref{tab:REAADM} se presenta la descripción de los requerimientos, que se estarán proporcionando como servicio de atención. 


  % Table generated by Excel2LaTeX from sheet 'REQUERIMIENTO'
\begin{table}[h!]
	\centering
	\caption{Categorización de requerimientos de procesos administrativos }
\scalebox{0.55}{	\begin{tabular}{|c|c|r|r|}
		\toprule
		\rowcolor[rgb]{ .267,  .447,  .769} \textcolor[rgb]{ 1,  1,  1}{Tipo de servicio} & \textcolor[rgb]{ 1,  1,  1}{Categoria} & \multicolumn{1}{c|}{\textcolor[rgb]{ 1,  1,  1}{Subcategoria}} & \multicolumn{1}{c|}{\textcolor[rgb]{ 1,  1,  1}{Servicios de IT}} \\
		\midrule
		\multirow{18}[36]{*}{\begin{turn}{-90}Requerimiento \end{turn}} & \multirow{18}[36]{*}{\begin{turn}{-90}Procesos Administrativo.\end{turn}} & \multicolumn{1}{r|}{\multirow{10}[20]{*}{Implementación de equipo de cómputo/ alta de equipo    }} & Configuración de dominio  \\
		\cmidrule{4-4}        &   &   & Configuración de red  \\
		\cmidrule{4-4}        &   &   & Configuración de perfil del usuario  \\
		\cmidrule{4-4}        &   &   & Configuración de Impresoras \\
		\cmidrule{4-4}        &   &   & Migración de información  \\
		\cmidrule{4-4}        &   &   & Configuración de carpetas compartidas \\
		\cmidrule{4-4}        &   &   & Configuración de PST  \\
		\cmidrule{4-4}        &   &   & Configuración de aplicativos  \\
		\cmidrule{4-4}        &   &   & Configuración de correo electrónico  \\
		\cmidrule{4-4}        &   &   & Creación de resguardo  \\
		\cmidrule{3-4}        &   & \multicolumn{1}{r|}{\multirow{3}[6]{*}{Borrado y retiro de equipo de cómputo / baja de equipo  }} & Borrado seguro  \\
		\cmidrule{4-4}        &   &   & Validación de certificado de borrado  \\
		\cmidrule{4-4}        &   &   & Baja de resguardo de equipo de computo  \\
		\cmidrule{3-4}        &   & \multicolumn{1}{r|}{\multirow{3}[6]{*}{Actualización de resguardo }} & Validación de componentes del equipo   \\
		\cmidrule{4-4}        &   &   & Actualización de datos del equipo  \\
		\cmidrule{4-4}        &   &   & Actualización de datos de usuario  \\
		\cmidrule{3-4}        &   & \multicolumn{1}{r|}{\multirow{2}[4]{*}{Reubicacion }} & Actulizacion de datos del usuario  \\
		\cmidrule{4-4}        &   &   & Actulizacion de Informacion de ubicación  \\
		\bottomrule
	\end{tabular}%
	\label{tab:REAADM}}%
\end{table}%



\subsubsection{Matriz de urgencia e impacto}

Dentro de las buenas prácticas de ITIL para la gestión de incidentes es necesario establecer una matriz de prioridad en función a la urgencia e impacto, cual  permita establecer tiempos de atención en los incidentes como se muestra en la tabla \ref{tab:MATURG}.
  % Table generated by Excel2LaTeX from sheet 'Hoja6'
\begin{table}[H]
	\centering
	\caption{Matriz de urgencia e impacto}
	\begin{tabular}{|c|c|p{31.145em}|}
		\toprule
		\rowcolor[rgb]{ .267,  .447,  .769} \multicolumn{3}{|c|}{\textcolor[rgb]{ 1,  1,  1}{Definición de impacto}} \\
		\midrule
		\rowcolor[rgb]{ .267,  .447,  .769} \textcolor[rgb]{ 1,  1,  1}{Nivel} & \textcolor[rgb]{ 1,  1,  1}{Valor} & \multicolumn{1}{c|}{\textcolor[rgb]{ 1,  1,  1}{Descripción}} \\
		\midrule
		Alto & Impacto 3 & El equipo o servicio no se encuentra disponible u trabaja con algunas restricciones perjudicando de manera masiva o colando en riesgo el servicio. Se atiende de forma prioritaria de acuerdo a los SLA pactados. \\
		\midrule
		mediano & Impacto 2 & El usuario no puede trabajar derivado del fallo de el equipo, sistema o aplicativo importante para la operación y finalización de un trabajo \\
		\midrule
		Bajo & Impacto 1 & El equipo, sistema o aplicativo trabaja con algunas restricciones. El Impacto es mínimo el usuario. El problema no manifiesta riesgo o impacto en la finalización de un trabajo. \\
		\bottomrule
	\end{tabular}%
	\label{tab:MATURG}%
\end{table}%



\subsubsection{Nivel de servicio}
El acuerdo de nivel de servicio (Service Level Agreement, SLA), es un documento resultante de la Gestión del Nivel de Servicio (de la Disciplina Diseño del Servicio), y representa el acuerdo entre un cliente y un proveedor de servicios de TI. Este SLA, especifica un servicio de TI, con sus objetivos de nivel de servicio y las responsabilidades del proveedor de servicios de TI y del cliente. Se debe comprender que el SLA es una extensión de los servicios del Catálogo de Servicios, ya que define principalmente las metas de atención de estos en base a una prioridad, grupo de cliente(s) a quien se ofrece el servicio y responsabilidades mutuas. En definitiva, el SLA se define desde el punto de vista del cliente que tenga que ser atendido.

Actualmente la empresa cuenta con dos tipos de SLA’s, el primer SLA referirá al primer contacto con el usuario, dando por consecuencia el segundo SLA, si es dado caso que no se pudiera atender el requerimiento o incidente según sea el caso, dentro del primero SLA, los SLA’s se estarán dado en horas como se muestra en la tabla \ref{tab:SLA1}.


  % Table generated by Excel2LaTeX from sheet 'Hoja7'
\begin{table}[H]
	\centering
	\caption{Niveles de SLA}

\scalebox{0.60}{	 \begin{tabular}{|r|r|r|r|r|}
	\toprule
	\rowcolor[rgb]{ .184,  .329,  .588} \textcolor[rgb]{ 1,  1,  1}{\textbf{Prioridad}} & \textcolor[rgb]{ 1,  1,  1}{\textbf{Tiempo de Respuesta}} & \textcolor[rgb]{ 1,  1,  1}{\textbf{Tiempo en solución (hrs)}} & \textcolor[rgb]{ 1,  1,  1}{\textbf{Tiempo Total (hrs)}} & \textcolor[rgb]{ 1,  1,  1}{\textbf{Servicio}} \\
	\midrule
	\multicolumn{1}{|c|}{\multirow{2}[4]{*}{Critico}} & \multicolumn{1}{c|}{\multirow{2}[4]{*}{2}} & \multicolumn{1}{c|}{\multirow{2}[4]{*}{16}} & \multicolumn{1}{c|}{\multirow{2}[4]{*}{18}} & \multicolumn{1}{p{19.855em}|}{Incidencia} \\
	\cmidrule{5-5}        &   &   &   & \multicolumn{1}{p{19.855em}|}{Requerimiento } \\
	\midrule
	\multicolumn{1}{|c|}{\multirow{2}[4]{*}{Alto}} & \multicolumn{1}{c|}{\multirow{2}[4]{*}{18}} & \multicolumn{1}{c|}{\multirow{2}[4]{*}{24}} & \multicolumn{1}{c|}{\multirow{2}[4]{*}{52}} & \multicolumn{1}{p{19.855em}|}{Incidencia} \\
	\cmidrule{5-5}        &   &   &   & \multicolumn{1}{p{19.855em}|}{Requerimiento } \\
	\midrule
	\multicolumn{1}{|c|}{\multirow{2}[4]{*}{Bajo}} & \multicolumn{1}{c|}{\multirow{2}[4]{*}{48}} & \multicolumn{1}{c|}{\multirow{2}[4]{*}{120}} & \multicolumn{1}{c|}{\multirow{2}[4]{*}{168}} & \multicolumn{1}{p{19.855em}|}{Incidencia} \\
	\cmidrule{5-5}        &   &   &   & \multicolumn{1}{p{19.855em}|}{Requerimiento } \\

	\bottomrule
\end{tabular}%

	\label{tab:SLA1}}%
\end{table}%


\subsubsection{Roles y puesto de mesa de servicio}

La organización en una empresa es necesaria e indispensable ya deben aportar a las personas y trabajadores la cultura de la empresa, detallando y clarificando las estrategias y los objetivos del negocio, para lo cual el personal debe tener una visión clara sobre su misión, rol, implicación y contribución dentro del proceso del servicio.

Para ITIL las personas son los elementos clave de una de sus dimensiones, donde  agrupa a todo el personal implicado en la entrega del servico y no únicamente la entidad de informática, por lo cual ITIL como marco de referencia sugiere para el desarrollo de una mesa de servicio pensando el personal que al momento de diseñar un servicio, se definan que responsabilidades tendrá dicho rol para determinar la autoridad adecuada, así mismo asignar identificar si el rol será asignado a uno o más personas. 

A continuación, una breve descripción del personal que forma parte en la atención y gestión de una Mesa de servicios.

\textbf{Service desk manager (gerente de mesa de servicio)}



El service desk manager se encarga de coordinar el equipo de la mesa de servicios, mantener la coordinación entre las partes involucradas. 

Un service desk manager es el responsable de que los servicios se entreguen de manera oportuna y también sirve como enlace de la mesa de servicio para la ejecución de las principales iniciativas que impactan el negocio.

Es el encargado de Supervisar el personal a su cargo y evaluará algunos aspectos como:

\begin{itemize}
	\item 	Evaluación de desempeño del personal.
\item 		Organizar y planificar las actividades con los agentes de cuentas.
	\item 	Cumplir los procedimientos de la Mesa de Servicios y asegurarse que su personal a cargo lo realice.
\item 		Realizar estadísticas de incidentes.
\item 		Dar seguimiento de las tareas asignadas a cada agente.
\item 		Administrar los incidentes, pedidos o reclamos recibidos sobre los servicios atendidos.
\item 		Emitir informe de servicios semanal y mensual.
\item 		Contribuir al desarrollo de los manuales de normas y procedimientos, detectar necesidades de capacitación de los miembros de su equipo.
	
\end{itemize}

\textbf{Coordinador de la mesa de servicio}

Es el encargado de:
\begin{itemize}
\item Supervisar las actividades incluidas en los servicios.
\item Asegurar el nivel de servicio, gestión y organización del equipo de trabajo.
\item Aplicar las mejores prácticas definidas por ITIL. 
\item Mantener una actitud proactiva frente a las oportunidades de mejora de los servicios.
\item Coordinar la realización de la encuesta de satisfacción del servicio.
\item Detectar necesidades de capacitación del personal.

\end{itemize}
\textbf{Agentes de la mesa de servicio}

Es el encargado de recibir llamadas o correos por los usuarios de los clientes de la empresa donde:


\begin{itemize}
\item Exista interrupción no planificada o reducción de la calidad del servicio
\item Interrumpa la operación normal de trabajo.
\item Requerimiento de soporte sobre el software y hardware a primer nivel.
\item Efectúen consultas planteadas por usuarios, distintos tipos de asesoramientos en el funcionamiento y utilización de los recursos informáticos.
\item Identificar los problemas, primera instancia de llamada 
\item Confirmar la satisfacción del usuario con respecto a la solución brindada.
\item Levantamiento de Incidencias.

\end{itemize}
\textbf{Técnicos de la mesa de servicio}

Es el encargado de:
\begin{itemize}
	\item Ejecutar trabajos de mantenimiento preventivo y correctivo de los equipos de computación.
	\item Brindar el soporte oportuno en los sistemas informáticos, que comprende lo siguiente
	\begin{itemize}
			\item Mantenimiento e implantación de software.
		\item Mantenimiento de base de datos de usuarios y correos.
		
	\end{itemize}

	\item Capacitar en el uso de herramientas.
	\item Documentar las soluciones dadas para mantener actualizado el Catálogo de Servicio.
	
\end{itemize}


Para realizar una mejor interpretación de los roles de trabajo,  se presenta un modelo RACI  donde se especifica la interacción que tiene cada uno de los roles a participar en el servicio, como se muestra en la tabla \ref{tab:RACI}.

  % Table generated by Excel2LaTeX from sheet 'Hoja8'
\begin{table}[htbp]
	\centering
	\caption{Modelo RACI de roles}
\scalebox{0.55}{	     
	\begin{tabular}{|p{15.355em}|p{11.215em}|p{13.645em}|p{17.285em}|p{14.355em}|}
		\toprule
		\rowcolor[rgb]{ .184,  .329,  .588} \multicolumn{1}{|r|}{\textcolor[rgb]{ 1,  1,  1}{}} & \multicolumn{1}{r|}{\textcolor[rgb]{ 1,  1,  1}{\textbf{Agentes de la mesa}}} & \multicolumn{1}{r|}{\textcolor[rgb]{ 1,  1,  1}{\textbf{Gerente de mesa}}} & \multicolumn{1}{r|}{\textcolor[rgb]{ 1,  1,  1}{\textbf{Coordinador de la mesa}}} & \multicolumn{1}{r|}{\textcolor[rgb]{ 1,  1,  1}{\textbf{Técnicos de la mesa }}} \\
		\midrule
		Apertura de Ticket & \textbf{C} & \multicolumn{1}{c|}{} & \textbf{I} & \multicolumn{1}{c|}{} \\
		\midrule
		Documentación de Incidente & \textbf{R} & \multicolumn{1}{c|}{} & \textbf{A} & \multicolumn{1}{c|}{} \\
		\midrule
		Asignación de Incidente & \textbf{R} & \textbf{A} & \textbf{R} & \textbf{C} \\
		\midrule
		Análisis de incidentes & \textbf{C} & \textbf{I} & \textbf{I} & \textbf{R/A} \\
		\midrule
		Escalamiento de incidente & \textbf{I} & \textbf{I} & \textbf{I} & \textbf{R/A} \\
		\midrule
		Cierre de ticket & \textbf{C} & \textbf{I} & \textbf{I} & \textbf{R/A} \\
		\bottomrule
	\end{tabular}%
	\label{tab:RACI}}%
\end{table}%

\section{Requerimientos funcionales}




\begin{itemize}
	\item Módulo de Registro y levantamiento de incidentes.
\end{itemize}