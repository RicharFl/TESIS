
%%%%%%%%%%%%%%%%%%%%%%%%%%%%%%%%%%%%%%%%%%%%%%%%%%%%%%%%%%%%%%%%%%%%%%%%%
%           Capítulo 3: NOMBRE                   %
%%%%%%%%%%%%%%%%%%%%%%%%%%%%%%%%%%%%%%%%%%%%%%%%%%%%%%%%%%%%%%%%%%%%%%%%%

\chapter{Marco Teórico}
\section{¿Qué es una mesa de Servicio?}
La mesa de Servicio comenzó como un sistema de soporte para solucionar problemas de TI. Fue un trabajo extremadamente técnico centrado en la tecnología en lugar de los usuarios finales. En los primeros días, los servicios de asistencia de TI no tenían que lidiar con ningún tipo de SLA para resolver problemas. No fue hasta ITIL entró en escena que definió y capturó las mejores prácticas de Gestión de Servicios de TI. El modelo de la mesa de servicio de TI centrada en el usuario comenzó a salir a la luz. La mesa de servicio fue vista como un componente necesario del manejo de TI.

Además, los servicios de TI se consideraron un sistema valioso que puede ofrecer respuestas rápidas y reactivas a los problemas del usuario. Y comenzó a ganar una posición única en la industria de TI. Fue utilizado para interactuar y comunicarse diariamente con los consumidores y los empleados. Los datos y las percepciones obtenidas de los problemas técnicos, las elecciones de los usuarios y lo que los usuarios contentos ahora comenzaron a considerarse valiosos para la configuración y el desarrollo de diferentes soluciones de TI. [13]
\subsection{Características de la mesa de servicio}

El principal objetivo de la mesa de servicio es garantizar la satisfacción del cliente. Para ello, se enfoca en evitar fallas, cubrir cuellos de botella y asegurar una prestación de servicios de calidad. Actuando de forma estratégica y preventiva. [14]

Las principales características y funciones de la mesa de servicio son:
\begin{itemize}
	\item 	Actuar como un único punto de contacto para todos los usuarios de los servicios de TI;
\item 	Restablecer el "funcionamiento normal del servicio" lo más rápido posible en caso de una interrupción;
\item 	Rastrear y categorizar preguntas y consultas para ayudar a los gerentes a predecir problemas;
\item 	Apoyar y guiar a la mesa de ayuda desde el principio hasta el final;
\item 	Actuar de forma proactiva para resolver solicitudes complejas de TI;
\item 	Administrar los ciclos de vida del programa, lo que permite un flujo constante de datos;
\item 	Realizar el mantenimiento de todos los sistemas y programas;
\item 	Estudiar e implementar nuevas herramientas tecnológicas que ayuden a asegurar el mejor desempeño de la empresa;
\item 	Administrar los permisos de acceso de los usuarios
\item 	Elaborar informes que muestren y monitoreen el avance del trabajo, verificando que esté alineado con los objetivos predefinidos. 
\end{itemize}

\section{Plataforma como Servicio (PaaS)}
Plataforma como servicio (PaaS) es un entorno de desarrollo e implementación completo en la nube, con recursos que permiten entregar todo, desde aplicaciones sencillas basadas en la nube hasta aplicaciones empresariales sofisticadas habilitadas para la nube. 

El cliente le compra los recursos que necesita a un proveedor de servicios en la nube, a los que accede a través de una conexión segura a Internet, pero solo paga por el uso que hace de ellos.

Al igual que IaaS, PaaS incluye infraestructura (servidores, almacenamiento y redes), pero también incluye middleware, herramientas de desarrollo, servicios de inteligencia empresarial (BI), sistemas de administración de bases de datos, etc. 

PaaS está diseñado para sustentar el ciclo de vida completo de las aplicaciones web: compilación, pruebas, implementación, administración y actualización. 

PaaS permite evitar el gasto y la complejidad que suponen la compra y la administración de licencias de software, la infraestructura de aplicaciones y el middleware subyacentes, los orquestadores de contenedores como Kubernetes, o las herramientas de desarrollo y otros recursos. 
El cliente de nube es el encargado de administrar las aplicaciones y los servicios que desarrolla y, normalmente, el proveedor de servicios en la nube administra todo lo demás. [15]

\subsection{Ventajas de PaaS }

\textbf{Reducir el tiempo de programación}

 Las herramientas de desarrollo de PaaS pueden reducir el tiempo que se tarda en programar aplicaciones nuevas con componentes de aplicación preprogramados que están integrados en la plataforma, como flujos de trabajo, servicios de directorio, características de seguridad, búsqueda, etc. 
 
 \textbf{Agregar más funcionalidad de desarrollo sin incorporar más personal}
  Los componentes de plataforma como servicio pueden aportar a su equipo de desarrollo nuevas características sin necesidad de contratar personal especializado. 
 Desarrollar para varias plataformas (incluidos los dispositivos móviles) con más facilidad. Algunos proveedores de servicios ofrecen opciones de desarrollo para varias plataformas, como PC, dispositivos móviles y exploradores, lo que agiliza y facilita el desarrollo de aplicaciones multiplataforma. 
 
 \textbf{Usar herramientas sofisticadas a un precio asequible}
  Gracias a un modelo de pago por uso, las personas u organizaciones pueden usar software de desarrollo sofisticado y herramientas de inteligencia empresarial y análisis cuya compra no se podrían permitir. 
\textbf{  Colaboración en equipos de desarrollo distribuidos geográficamente}
  Puesto que al entorno de desarrollo se accede a través de Internet, los equipos de desarrollo pueden colaborar en proyectos incluso si los miembros del equipo se encuentran en lugares diferentes. 

 \textbf{Administrar el ciclo de vida de las aplicaciones con eficacia}
  PaaS proporciona todas las características necesarias para sustentar el ciclo de vida completo de las aplicaciones web: compilación, pruebas, implementación, administración y actualización, dentro del mismo entorno integrado. [16]
 
 \section{Servicios Web}
 Los servicios web son aplicaciones autónomas modulares que se pueden describir, publicar, localizar e invocar a través de una red.
 El servidor de aplicaciones da soporte a los servicios web que se desarrollan e implementan de acuerdo con la especificación de servicios web para Java™ EE (Java Platform, Enterprise Edition). El servidor de aplicaciones da soporte a los modelos de programación JAX-WS (Java API for XML Web Services) y JAX-RPC (Java API for XML-based RPC ). JAX-WS es un modelo de programación estratégico que simplifica el desarrollo de aplicaciones mediante el soporte de un modelo estándar basado en anotaciones para desarrollar clientes y aplicaciones de servicios web. [15]
 
 \section{ ITIL }
 
 \subsection{Evolución de ITIL.}
\textbf{Primera versión de ITIL.}

 En su primera versión a finales de la década de 1980, el enfoque inicial de ITIL como marco de trajo fue asegurar que la infraestructura instalada de equipos de cómputo en las organizaciones operara correctamente y con nulo impacto operativo.
 
\textbf{Segunda versión de ITIL V2.}

 Hacia el año 2004 la segunda versión ITIL se centra en promover en las empresas la necesidad de contar con un área de tecnología de información con la asignación de un presupuesto que apoya sus procesos internos con los equipos de cómputo personal y el software para lograr el objetivo de duchas empresas.
 
\textbf{Tercera versión de ITIL V3.}

 La tercera versión de ITIL difundida en 2007 tuvo como enfoque principal mejorar la relación con el cliente y los proveedores del servicio, así como ayudar a las organizaciones en su constante actualización con el apoyo de nuevas tecnologías para obtener mejoras con respecto a ITIL V.2., finalmente se integró la mejora continua como una práctica organizacional, haciendo enfoque a la mejora de procesos, ya establecidos en el marco de trabajo de ITIL y cabe señalar que ITIL V3 tuvo una modificación menor en su proceso en el año 2011.
 
\textbf{Cuarta versión de ITIL V4.}

 Hoy en día y ITIL 4 se enfoca en ofrecer las mejores prácticas en la entrega de servicios considerando las tecnologías y los conceptos de las últimas generaciones en lo que se conoce como transformación digital adaptando nuevas formas de trabajo por ejemplo marcos de trabajo ágiles o de DEVOPS.
 ITIL 4 descansa sobre 2 pilares para por encima de lo que lo que se construyen sus propuestas para una mejor entrega de servicios de tecnologías de la información estas son:
 \begin{itemize}
 	\item El sistema de valor de servicio útil
 	\item	El modelo de cuatro dimensiones de gestión de servicio
 \end{itemize}
 
 A través del sistema de valor de servicios ITIL las organizaciones determinan cómo utilizar los recursos activos y capacidades para que a través del tiempo generen valor a sus clientes en este aspecto es posible decir que el sistema de valor de servicio ITIL cuenta con 5 componentes.
 \begin{enumerate}
 	\item Principio de guía: son recomendaciones que pueden guiar a una organización en cualquier circunstancia independiente de los cambios en su objetivo estrategias tipo de trabajo o estructura de gestión
 	\item	Gobernabilidad:  son los medios por los cual es una organización es dirigida y controlada
 \item Cadena de valor de servicio: es un modelo operativo que describe las actividades clave necesarias para responder a la demanda y facilitar la realización de valor a través de la creación de gestión de productos y servicios.
 	\item	Prácticas: es decir un conjunto de recursos organizacionales diseñados para realizar un trabajo o lograr
 	\item	Mejora continua: es una estrategia organizacional que tiene como objetivo mejorar los productos, servicios, procesos operativos y las relaciones de la organización.
 \end{enumerate}
 
 
 Para conocer mas el marco de trabajo ITIL, se definen lo que para ITIL son conceptos importantes, los  cuales a lo largo del documento se estarán utilizando con la definición propuesta por la misma.
 
 \subsection{¿Qué es un Servicio?}
 Un servicio dentro de la ITIL es el medio que permite la creación conjunta de valor al facilitar los resultados que el cliente desea lograr, sin que éste tenga que administrar costos y riesgos específicos, en otras palabras, a través de los recursos y capacidades se trabaja con los clientes de manera conjunta para determinar la forma en que se resolverán sus problemas y se generará valor en la medida en que se especifica cuál es la utilidad y la garantía de los servicios tienen que cumplir.
 \subsection{¿Qué es la utilidad?}
 La utilidad según ITIL es lo que es un servicio para mejorar el desempeño de un cliente Y/O para eliminar una restricción de un negocio
  \subsection{¿Que son las garantías?}
 Al hablar de garantía se hace referencia a elementos de desempeño tales como.
 \begin{itemize}
 	\item  Capacidad: son las características que tiene un servicio para que un usuario lo pueda utilizar según lo acordado en tiempo y lugar
 	\item Disponibilidad: son las características que tiene un servicio para atender o afrontar adecuadamente la demanda
	\item Seguridad: hace referencia a la protección de los servicios de infraestructura y la información del cliente contra ataques informáticos comprende la integridad la confidencialidad y la durabilidad de la información, así como también integra los procesos de automatización entre dispositivos
	\item continuidad de los servicios: está relacionado con las características que tiene el mismo para que pueda sobrevivir parcial o totalmente encontré un chingo de lana en una pierna
 \end{itemize}


\section{Ciclo de vida del servicio ITIL}
 
 ITIL en su ciclo de vida propone múltiples conceptos, estos a su vez  basándose en el  ciclo de vida del servicio, así mismo incluyendo esencialmente la “Gestión del Servicio” y los conceptos relacionados de “Servicio” y “Valor”. 
 \subsection{Estrategia del Servicio (Service Strategy)}
 las estrategias de servicio proporcionar una guía, tanto a los proveedores de servicios como a sus clientes, con la intención de ayudarles a operar y prosperar, mediante el establecimiento de una estrategia de negocio bien definida.
 Esta fase de estrategia es donde la alta gerencia da luz verde para que inicie un servicio, es decir autorizan si va o no a realizarse un servicio,  por lo cual  en esta fase solo se encuentran a las altas gerencias y mandos medios para la toma de decisiones. [19]
 

 Esta fase incluye a los procesos siguientes:
 
 \begin{itemize}
 	\item Gestión de Portafolio de Servicios (SPM): En este proceso encontramos a las altas gerencia y mandos medios, esta gestión es netamente administrar los servicios existentes del negocio y tener un histórico de los servicios antiguos. El gerente tiene  visión de sus servicios y conoce la naturaleza de los mismos. 
 	\item 	Gestión Financiera: Agrupa los procesos y actividades asociados con las finanzas de la Gestión del Servicio. Entrega información de gestión indispensable para una operación eficiente y rentable
 \item  Gestión de la Demanda: son los procesos y actividades fundamentales para la gestión del servicio. ya que  permiten determinar la mejor asignación de recursos y  adquisición de artículos.
 	
 \end{itemize}

\subsection{Diseño de Servicio (Service Design, SD)}

 
El diseño de servicios ofrecer pautas para el diseño de servicios apropiados e innovadores, incluyendo su arquitectura, procesos, políticas y documentación, así mismo satisfacer los requisitos de negocio, actuales y futuros, acordados.
 
El objetivo principal es: El diseño de servicios nuevos o modificados, donde se define el alcance para su paso a un entorno de producción.[19]
 
 Los procesos de Diseño del Servicio son: 
 
  
 \begin{itemize}
 	\item \textbf{Gestión del Catálogo de Servicios (SCM)}: El objetivo general es el desarrollo y mantenimiento de un catálogo de servicios que incluya todos los datos precisos y el estado de todos los servicios existentes y de los procesos de negocio a los que apoyan, así como aquellos en desarrollo. 
 	\item  \textbf{Gestión de Niveles de Servicio (SLM):} El objetivo general de este proceso es garantizar que se cumplen los niveles de provisión de los servicios de TI, tanto existentes como futuros, de acuerdo con los objetivos acordados. 
 	\item \textbf{Gestión de la Capacidad}: El objetivo general de este proceso es garantizar que la capacidad se corresponde con las necesidades presentes y futuras del cliente (documentadas en un plan de capacidad).
 	\item  \textbf{	Gestión de la Disponibilidad}: El objetivo general de este proceso es garantizar que los niveles de disponibilidad de los servicios, nuevos o modificados, se corresponden con los niveles acordados con el cliente. Debe mantener el Sistema de Información de gestión de la disponibilidad (AMIS), que es la base del plan de disponibilidad. 
 	 \item\textbf{ Gestión de la Continuidad de los Servicios de TI (ITSCM)}: El objetivo es facilitar la continuidad del negocio (funciones vitales de negocio) garantizando la recuperación de las instalaciones de TI necesarias en el tiempo acordado. 
 	 \item \textbf{Gestión de la Seguridad de la Información}: Garantiza que la política de seguridad de la información satisface los requisitos generales de la organización, así como los que tienen su origen en el gobierno corporativo.
 	\item \textbf{ Gestión de Proveedores}: Este proceso se centra en todos los proveedores y contratos para facilitar la provisión de servicios al cliente.
 	
 \end{itemize}
\subsection{Transición del Servicio (Service Transition, ST)}
En esta fase la operación del servicio cubre la coordinación y ejecución de las actividades y procesos necesarios para entregar y gestionar servicios para usuarios y clientes, con el nivel de servicio acordado. La operación del servicio también tiene la responsabilidad de gestionar la tecnología necesaria para la prestación y el soporte de los servicios.[19]

Los procesos de Transición del Servicio son:
\begin{itemize}
	\item \textbf{Gestión de la Configuración y Activos del Servicio (SACM)}: Gestiona los activos del servicio y elementos de configuración (CIs) para dar soporte a los demás procesos de Gestión del Servicio.
	\item	\textbf{Gestión de Cambios}: Garantiza que los cambios se aplican de una manera controlada,  evaluados, priorizados, planificados, probados, implementados y documentados. 
\item	\textbf{Evaluación del Cambio}: Es un proceso genérico cuyo objetivo consiste en verificar si el rendimiento de algo es aceptable; por ejemplo, si tiene una buena relación calidad/precio, si es continuo, si está en uso, si hay que pagar por ello, etc. 
	\item\textbf{	Gestión de Entregas y Despliegues}: Concentrado en construir, probar y desplegar los servicios especificados en el Diseño del Servicio, y en garantizar que el cliente/usuario puede utilizar el servicio de manera efectiva.
	\item\textbf{	Validación y Pruebas del Servicio}: Las pruebas garantizan que los servicios nuevos o modificados están “ajustados al propósito” y “ajustados al uso”. 
	\item	\textbf{Gestión del Conocimiento}: Mejora la calidad de la toma de decisiones (de la dirección) garantizando la disponibilidad de información segura y fiable durante el Ciclo de Vida del Servicio.
	\item\textbf{ Planificación y Soporte de la Transición}: Garantiza que los recursos se planifican y coordinan adecuadamente para cumplir las especificaciones del Diseño del Servicio.
	
\end{itemize}

\subsection{Operación de Servicio (Service Operation, SO).}

En esta fase la operación del servicio cubre la coordinación y ejecución de las actividades y procesos necesarios para entregar y gestionar los servicios para usuarios y clientes, con el nivel de servicio acordado. La operación del servicio también tiene la responsabilidad de gestionar la tecnología necesaria para la prestación y el soporte de los servicios, esto lo  encontramos en el día a día efectuando la atención a los requerimientos de los clientes y/o usuarios. [19]

Los procesos de Operación del Servicio son: 

\begin{itemize}
	\item \textbf{Gestión de Peticiones}: Se encarga del tratamiento de peticiones de servicio de los usuarios, proporcionando un canal de solicitud, información y ejecución de la petición. 
\item \textbf{Gestión de Incidencias}: Se concentra en restaurar el fallo del servicio lo antes posible para los usuarios, de manera que su impacto sobre el negocio sea mínimo. 
\item\textbf{ Gestión de Problemas}: Incluye todas las actividades necesarias para diagnosticar las causas subyacentes de las incidencias y para encontrar una solución a esos problemas. 
\item \textbf{Gestión de Eventos}: Supervisa todos los eventos que se producen en la infraestructura de TI con el fin de monitorizar el rendimiento. Este proceso puede estar automatizado para efectuar un seguimiento y escalado ante circunstancias imprevistas.
\item \textbf{Gestión de Accesos}: Permite utilizar el servicio a los usuarios autorizados y limita el acceso a los usuarios sin autorización.
	
\end{itemize}


\subsection{Mejora Continua del Servicio (Continuous Service Improvement, CSI).}

Los departamentos de TI tienen que mejorar continuamente sus servicios para seguir atendiendo al llamamiento del negocio. De esto se ocupa la fase de Mejora Continua del Servicio (CSI) del ciclo de vida.
Se debería aplicar CSI a lo largo de todo el ciclo de vida, en todas sus fases, desde la Estrategia a la Operación. En este sentido, se convierte en algo inherente tanto al desarrollo como a la provisión de servicios de TI.[19]
