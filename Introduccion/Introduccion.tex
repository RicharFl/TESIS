% Thesis Abstract -----------------------------------------------------
%\begin{abstractslong}    %uncommenting this line, gives a different abstract heading
\chapter*{Introducción}
Los sistemas informáticos para la gestión de empresas se han convertido en una herramienta clave en el desarrollo empresarial, necesarias para cumplir los objetivos de los negocios y brindar atención a sus clientes,  Actualmente el mercado ofrece una variedad de herramientas para optimizar los procesos de las PYMES por medio de sistemas de informáticos que capturan, almacenan, procesan y distribuyen la información generada por las distintas unidades administrativas, operativas y productivas de las empresas.\\

Al menos en México existen más de 8 softwares capaces de realizar las actividades previamente mencionadas, pero con un costo elevado [1], Es así como los Sistemas de Planificación de Recursos Empresariales ERP (Enterprise Resource Planning) surgen de la necesidad de integrar todos los datos de una organización, permitiendo obtener información confiable y en tiempo real, desde la fabricación de un producto, pasando por la logística, la distribución, el control de stock, la contabilidad de la organización y demás.\\

En la actualidad la empresa PyME (solicitante del desarrollo de la mesa de servicio) basa su funcionamiento en ERP, sin embargo, requiere una solución que dé respuesta a la gestión, operación y administración de los incidentes que se generan, esta solución deberá adaptarse a la solución ERP con la que cuenta la PyME actualmente.
Nos referiremos a la empresa solicitante de la mesa de servicio como PyME para englobar de manera genérica las características de esta y evitaremos de esta forma utilizar el nombre que por razones de confidencialidad no estamos en posibilidades de mencionarla. Esta empresa tiene más de 8 años dedicada a brindar soluciones integrales de Telecomunicaciones y Administración, hoy cuenta con tres líneas de negocio: en una primera línea se encuentra el proveer capital humano para implementar equipos de cómputo a dependencias Gubernamentales, entenderemos por “implementar equipos de cómputo” como la actividad de cambiar un equipo de cómputo viejo a un nuevo, esta actividad  conlleva la entrega, configuración y documentación pertinente que evalúa el buen funcionamiento del nuevo equipo, en una segunda vertiente se encuentra el soporte técnico a dichos equipos, una vez llevada la fase de implementación se considera un periodo de soporte técnico por un lapso de tiempo el cual esta defino bajo contrato, por lo general estos proyectos constan de 3 años y finalmente una tercera ramificación se encuentra el control de inventarios de equipos de cómputo así como equipos de TI. \\

Por el crecimiento acelerado que ha tenido en los 3 últimos años, el incremento de incidentes ha aumentado de manera acelerada, por lo cual en este momento no cuenta con procesos definidos para la gestión de servicios de incidencias, así mismo carece de un correcto historial de requerimientos, incidentes y cortes de servicio en general. La falta de estos procedimientos repercute en la pérdida de tiempo, recursos e imagen de la PyME.\\

A lo largo de este documento se prende dar una solución integral a la problemática ya expuesta, desarrollando una mesa de servicios bajo las metodologías ITIL, dicha metodología será la base para el desarrollo de una plataforma como servicio (PaaS) que provea una mesa de servicios WEB.
\newpage
