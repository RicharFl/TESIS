
% Thesis Abstract -----------------------------------------------------


%\begin{abstractslong}    %uncommenting this line, gives a different abstract heading
\begin{abstracts}        %this creates the heading for the abstract page

El continuo crecimiento y desarrollo de las Pequeñas y Medianas Empresas (PyME) dedicadas al soporte técnico e inventarios de infraestructuras de las Tecnologías de la información (TI) de la  Ciudad de México, las ha impulsado a revisar e implementar procesos o métodos de especialización sobre necesidades específicas, que les permitirán marcar una diferencia ante sus competidores.
\newline
El siguiente proyecto presenta como objetivo emplear la guía de las buenas prácticas para la gestión de servicios de tecnologías de la información (ITIL). Esto, a través de la implementación de una mesa de servicio especifica a la necesidad del cliente, la cual sea capaz de gestionar adecuadamente los incidentes y requerimientos que puedan surgir. 
\newline
Para lo anterior, se ha propuesto una aplicación web alojada en la nube mediante la plataforma como servicio (PaaS), complementándose a través de las tecnologías como son el Lenguaje de Consulta Estructurada (SQL), HyperText Markup Language (HTML), Cascading Style Sheets (CSS), JavaScript con su framework Angular.js y Java con su framework Spring Boot. 
\newline
Finalmente, esta tesis pretende ser un aporte para todos aquellos lectores interesados en las vastas herramientas de la informática y quienes además tienen el privilegio de transitar en los pasillos de este honorable Instituto Politécnico Nacional (IPN). 

\begin{center}
	\textbf{Palabras clave: mesa de servicio, PaaS, buenas prácticas ITIL}
\end{center}

\begin{flushright}
	\textbf{\textit{Abstract}}
\end{flushright}

The continuous growth and development of Small and Medium Enterprises (PyME) dedicated to technical support and information technology (IT) infrastructure inventories in Mexico City, has prompted them to review and implement processes or methods of specialization on specific needs, which will allow them to make a difference to their competitors.
\newline
The following project aims to use the best practices guide for information technology service management (ITIL). This, through the implementation of a service desk specific to the client's needs, which is able to adequately manage incidents and requirements that may arise. 
\newline
For the above, it has been proposed a web application hosted in the cloud through the platform as a service (PaaS), complemented through technologies such as Structured Query Language (SQL), HyperText Markup Language (HTML), Cascading Style Sheets (CSS), JavaScript with its framework Angular.js and Java with its Spring Boot framework. 
\newline
Finally, this thesis aims to be a contribution to all those readers interested in the vast tools of computer science and who also have the privilege of walking in the corridors of this honorable Instituto Politécnico Nacional (IPN).

\textbf{\begin{center}
		Keywords: service desk, PaaS, Best Practices ITIL.
\end{center}}

\end{abstracts}
%\end{abstractlongs}


% ----------------------------------------------------------------------